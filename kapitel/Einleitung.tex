\section{Introduction}
Software Decay describes the gradual deterioration of software quality and maintainability, which creates significant challenges across various software domains.
The challenges become even more pronounced in environments where the systems are expected to operate for decades, such as in a military context.
The motivation for this thesis stems from observations of the difficulties of managing aging software projects, particularly in the military domain.
Over time the systems become increasingly problematic to maintain and modify, which highlights the importance of proactively maintaining the software quality and thereby ensuring long-term viability.\\
The long system lifecycle is very common in military software projects, which need to be robust and dependable, while also having regulatory requirements,
high security standards and complex stakeholder demands. With these unique factors in mind, usual software engineering practices to mitigate software decay 
could face limitations and require certain adaptations. Therefore, this thesis specifically investigates how software decay can be prevented or slowed down
with the specific requirements of the German military sector.\\
The main research question of this thesis is:
\begin{quote}
    \textit{How can software decay in military systems be prevented or slowed down?}
\end{quote}
To answer this overarching question, the following sub-questions have been formulated:
\begin{itemize}
    \item What are the main causes and symptoms of software decay specific to the military software domain?
    \item Which mitigation strategies are currently in use at T-Systems IFS, Chapter Defense Systems, and how effective are they?
\end{itemize}
The main goal of this thesis is to identify the specific constraints influencing software decay within military software projects and to evaluate existing practices that could effectively mitigate the decay.
Through a literature review and semi-structured expert interviews conducted with three practitioners from T-Systems IFS, Chapter Defense Systems, the thesis uses both theoretical and practical insights to answer the research questions.\\

The thesis is structured as follows: Initially a literature review is conducted to establish the theoretical foundation. This includes general information about software decay and technical debt,
as well as common mitigation strategies and unique constraints in the military domain. The second part of the thesis presents the methodology and explains the research design and analysis of the expert interviews.
In the third part, the findings of the interviews are presented in key themes derived from the data. These findings are further interpreted and discussed in the fourth part of the thesis, which integrates the theoretical insights with the practical findings.
Concluding, the final chapter synthesizes the key findings of the research, highlighting the implications for practitioners, as well as limitations and future research opportunities.\\

The primary audiences for this thesis include industry practitioners in the military software domain, as well as stakeholders from military institutions seeking to understand and improve their long-term software management practices.