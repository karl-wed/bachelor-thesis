\section{Literature Review}
\subsection{Problems of Software Engineering}
Software has become an integral part of our daily lives. Certain expectations exist regarding the quality of software in terms of reliability, security and efficiency. These expectations come with challenges for the software developers across all industries.
For decades, software engineers have tried to develop methods and guides to overcome these issues. However, in 1986 Frederick Brooks published his paper `No Silver Bullet' \footcite{brooksNoSilverBullet1987} 
in which he argues that: `\ldots building software will always be hard. There is inherently no silver bullet.' \footcite[3]{brooksNoSilverBullet1987}. He based this statement on the fact that there two types of difficulties in software development: the essential and the accidental.
The essential difficulties he names are complexity, conformity, changeability and invisibility.\\
With complexity, Brooks wants to describe the inherit intricacy of software systems: `Software entities are more complex for their size than perhaps any other human construct, because no two parts are alike'.\footcite[3]{brooksNoSilverBullet1987}
This complexity makes `conceiving, describing, and testing them hard'\footcite[3]{brooksNoSilverBullet1987}.\\
The second essential difficulty Brooks names is conformity. To explain this, he compares software development to physics. Even though they are similarly complex, physics has the advantage of relying on a single set of laws or `creator'. The same cannot be said for software engineers. Brooks claims that
the complexity is `arbitrary [\ldots], forced without rhyme or reason by the many human institutions and systems to which his interfaces must conform'\footcite[4]{brooksNoSilverBullet1987}. This is due to software being perceived as `the most comfortable'\footcite[4]{brooksNoSilverBullet1987} element to change in a system.\\
Brooks explains the third issue, changeability, by comparing software to other products like cars or computers. With these types of products, changes are difficult to make once the product is released. Software however is just `pure thought-stuff, infinitely malleable.'\footcite[4]{brooksNoSilverBullet1987} Another major issue regarding changeability is
the fact that software often `survives beyond the normal life of the machine vehicle for which it is first written'\footcite[4]{brooksNoSilverBullet1987}. This means that software has to be adapted to new machines causing an extended life time of the software.\\
Invisibility is the last essential difficulty Brooks names. With this he means the difficulty to visualize software compared to other products. This not only makes the creation difficult but also `severely hinders communication among minds'\footcite[4]{brooksNoSilverBullet1987}.
According to Brooks, these issues are in `the very nature of software' \footcite[2]{brooksNoSilverBullet1987}. These difficulties are unlikely to be solved, unlike the accidental difficulties.\\

In contrast, the accidental difficulties arise from limitations of current languages, tools and methodologies. According to Brooks, this involves issues such as inefficient programming environments, suboptimal development processes and integration challenges which can be overcome as the industry improves its practices and technologies.\footcite[5-6]{brooksNoSilverBullet1987}
For example, the adaptation of agile methodologies, integrated development environments and continuous integration have helped to overcome some of these accidental difficulties.\\

The persistent nature of these challenges presented by Brooks have since been substantiated by further empirical research. For instance, Lehman and Ramil (2003)\footcite{lehmanSoftwareEvolutionBackground2003} discussed in their paper that software systems that are left unchecked will experience a decline in quality over time.
This phenomenon is encapsulated in Lehman's laws of software evolution, which he formulated in multiple papers. In their paper `Software evolution - Background, theory, practice' Lehman and Ramil present empirical observations that support the notion that software quality tends to deteriorate over time - a phenomenon often described als
software decay. 

\subsection{Software Decay and Technical Debt}
The term software decay was empirically studied and statistically validated by Eick et al.\ in their influential paper `Does Code Decay? Assessing the Evidence from Change Management Data'(2001)\footcite{eickDoesCodeDecay2001}. They begin by stating that `software does not age or "wear out" in the conventional sense.' \footcite[1]{eickDoesCodeDecay2001}
If nothing in the environment changes, the software could run forever. However, this is almost never the case as mainly two things change constantly: the hard- and software environments and the requirements of the software.\\

This is in accordance with the first two laws of Program Evolution Dynamics formulated by Belady and Lehman (1976).
The first law states: `A system that is used undergoes continuing change until it is judged more cost effective to freeze and recreate it.'\footcite[228]{beladyModelLargeProgram1976}
Building on this, their second law suggests: `The entropy of a system (its unstructuredness) increases with time, unless specific work is executed to maintain or reduce it.'\footcite[228]{beladyModelLargeProgram1976}\\

Eick et al.\ analysis provides empirical validation for these theoretical laws, offering `very strong evidence that code does decay.'\footcite[7]{eickDoesCodeDecay2001}
They base this conclusion on their findings that `the span of changes increases over time'\footcite[7]{eickDoesCodeDecay2001} meaning that modifications to the software tend to affect increasingly larger parts of the system as the software evolves. This growth in the span of changes indicates - and potentially leads to -
a breakdown in the software's modularity. Consequently the software becomes `more difficult to change than it should be,'\footcite[3]{eickDoesCodeDecay2001} measured specifically by three criteria: cost of change, time to implement change and the resulting quality of the software.
Therefore, the combination of theoretical insights from Lehman and Belday and empirical data from Eick et al. paints a clear picture: software decay is an inevitable consequence of ongoing evolution unless consciously and proactively managed through structured efforts such as continuous refactoring and architectural vigilance.\\

The concept of software decay aligns closely with earlier theoretical discussions by David Parnas (1994). In his influential paper `Software Aging' Parnas describes software aging as a progressive deterioration of a program's internal quality primarily due to frequent, 
inadequately documented modifications which he termed `Ignorant surgery'\footcite[280]{296790}, as well as the failure to continuously adapt the architecture to evolving needs which he called `Lack of movement'\footcite[280]{296790}.
Without this proactive maintenance and refactoring effort, Parnas argues that software inevitably reaches a state where changes become more riskier, more costly and error-prone\footcite[280-281]{296790}.

//TODO: Emperical studies like Banker or Bieman maybe?

The term `Technical Debt' was first coined by Ward Cunningham in his paper `The WyCash Portfolio Management System' (1992)\footcite{cunninghamWyCashPortfolioManagement1992} This metaphor was used to describe the trade-off between a quickly implemented solution and a thought out process. 
When ones uses the quick solution it `is like going into debt.'\footcite[2]{cunninghamWyCashPortfolioManagement1992} Cunningham argues that this debt accumulates interest if not repaid or rewritten. 
If this does not happen Cunningham warns that `Entire engineering organizations can be brought to a stand-still under the debt load of an unconsolidated implementation'\footcite[2]{cunninghamWyCashPortfolioManagement1992}.\\

This term was further built upon and refined by the industry through white papers like `Technical Debt' by Steve McConnell (2008)\footcite{mcconnellManagingTechnicalDebt2017} or the `Technical Debt Quadrant' by Martin Fowler (2009)\footcite{fowlerTechnicalDebtQuadrant2009}.
McConnell differentiates between two types of technical debt: Unintentional and Intentional \footcite[3]{mcconnellManagingTechnicalDebt2017}. The first results from bad code, inexperience or unknowingly taking over a project with technical debt.
The second type is taken on purpose `to optimize for the present rather than for the future.'\footcite[3]{mcconnellManagingTechnicalDebt2017} As the first is not planned, it is difficult to avoid it. The second type however can be managed and controlled.\\
Additionally, McConnell differentiates between different types of intentional debt. According to him, debt can be taken on short-term or long-term. The short-term debt is taken on to meet a deadline or to deliver a feature. Therefore it is `taken on tactically or reactively'\footcite[3]{mcconnellManagingTechnicalDebt2017}.
The long-term debt on the other hand is more strategic and is taken on to help the team in a bigger picture. The difference between those two is that short-term debt `should be paid of quickly, perhaps as the first part of the next release cycle'\footcite[4]{mcconnellManagingTechnicalDebt2017}, while 
long-term debt is something companies can be carried for years.\\
Martin Fowler on the other hand warned in taking on too much deliberate debt. He argues that `Even the best teams will have debt to deal with as a project goes on - even more reason not to overload it with crummy code.'\footcite{fowlerTechnicalDebtQuadrant2009}
He created a quadrant between reckless and prudent and deliberate and inadvertent. The difference between reckless and prudent for Fowler is the way the debt is taken on. Reckless debt happens without the right evaluation of the consequences, risking difficulties in the future. Prudent on the other hand is taken on
with the trade-offs in mind and the knowledge of the future costs. Fowler differentiates between deliberate and inadvertent is similar to McConnell's differentiation between intentional and unintentional debt.
The combination of those four result in either quick solutions without considering the long-term impact (reckless and deliberate), flawed design or implementation, either carelessly or unknowingly (reckless and inadvertent), 
purposefully taking on debt to gain a short-term advantage with plans of repayment (prudent and deliberate) or taking on debt due to lack of knowledge or experience (prudent and inadvertent).\\

In their article `Technical Debt: From Metaphor to Theory and Practice' (2012) Kruchten et al.\ \footcite{kruchtenTechnicalDebtMetaphor2012} criticize the concept of technical to be `somewhat diluted lately' \footcite[18]{kruchtenTechnicalDebtMetaphor2012}, stating that every issue in software development was called some form of debt. 
Therefore they set out do define `a theoretical foundation'\footcite[19]{kruchtenTechnicalDebtMetaphor2012} for technical debt.\\
Kruchten et al.\ state that technical debt has become more than the initial coding shortcuts and rather encompasses all kinds of internal software quality comprises.\footcite[19]{kruchtenTechnicalDebtMetaphor2012}
According to them this includes architectural debt, `documentation and testing'\footcite[20]{kruchtenTechnicalDebtMetaphor2012} as well as requirements and infrastructure debt.
All these debt types allow engineers to better discuss the trade-offs with stakeholders and to make better decisions.\\
Additionally, Kruchten et al. build upon the metaphor of Cunnigham 
\\TODO: More about Kruchten and Theory?

There have been many studies providing empirical evidence to the theoretical concepts of technical debt. Two very influential ones are the study by Potdar and Shihab (2014) \footcite{potdarExploratoryStudySelfAdmitted2014} as well as the study by Li et al.\ 2015 \footcite{liSystematicMappingStudy2015}.\\
To do this Potdar and Shihab analyzed four large open source projects to find self admitted technical debt as well as how likely it is that the debt will be removed. They found that `self-admitted technical debt exists in 2.4\% to 31\% of the files.'\footcite[1]{potdarExploratoryStudySelfAdmitted2014}
Additionally, they found that `developers with higher experience tend to introduce most of the self-admitted technical debt and that time pressures and complexity of the code do not correlate with teh amount of the self-admitted technical debt.'\footcite[1]{potdarExploratoryStudySelfAdmitted2014}
They also discovered, that `only between 26.3\% and 63.5\% of the self-admitted technical debt gets removed'\footcite[1]{potdarExploratoryStudySelfAdmitted2014}. This relatively low removal rate of self-admitted technical debt indicates a wider challenge:
developers recognize the issues of their implementation, but defer remediation potentially leading to a major impact on long-term maintainability.\\
Another approach to provide empirical evidence towards technical debt was taken by Li et al.\ . They conducted a systematic mapping study to `get a comprehensive understanding of the concept of "technical debt"'\footcite[194]{liSystematicMappingStudy2015}, as well as getting an overview of the current research in the field.
Different questions where asked like what types of technical debt exists, how does technical debt affect software quality and what is the limit of the technical debt metaphor.\\
They found that the `10 types of TD are requirements TD, architectural TD, design TD, code TD, test TD, build TD, documentation TD, infrastructure TD, versioning TD, and defect TD.'\footcite[215]{liSystematicMappingStudy2015}
Additionally they found that `Most studies argue that TD negatively affects the maintainability [\ldots] while other QAs and sub-QAs are only mentioned in a handful of studies'\footcite[215]{liSystematicMappingStudy2015}.\\
During their studies,  Li et al. observed that the inconsistent and arbitrary use of the term `debt' among researchers and practitioners can cause confusion and hinder effective management of technical debt.\footcite[211]{liSystematicMappingStudy2015} Additionally practitioners `tend to connect any software quality issue to debt, such as code smells debt, dependency debt and usability debt.'\footcite[212]{liSystematicMappingStudy2015}
This indicates a inflationary use of the term, which one has to be aware of when speaking about technical debt.\\

The implications these studies have on the software industry are significant. They show that software decay and technical debt are tangible and measurable in real world software projects.
In their paper `Software complexity and maintenance costs' \footcite{bankerSoftwareComplexityMaintenance1993} (1993) Banker et al. empirically demonstrated that `software maintenance costs are significantly affected by the levels of existing software complexity.' \footcite[12]{bankerSoftwareComplexityMaintenance1993}
This finding emphasizes the important of proactively managing the software quality and addressing debt early in the project lifecycle, to keep the complexity and therefore cost to a minimum.\\
To address these effects, practitioners strongly recommend refactoring. Fowler argued in his book `Refactoring: Improving the Design of Existing Code' (2019)
that `Without refactoring, the internal design - the architecture - of software tend to decay.'\footcite[58]{fowlerRefactoringImprovingDesign2019}
To prevent this he suggests `Regular refactoring [to] help[s] keep the code in shape'\footcite[58]{fowlerRefactoringImprovingDesign2019}.\\
To prevent long-term issues, practitioners recommend to actively manage technical debt through refactoring, tracking and other strategies 
that integrate debt management into the software development process.\\

\subsection{Mitigation Strategies in Commercial Environments}
Technical debt and software decay have been recognized as significant challenges in the software industry. They can lead to 
increased maintenance costs, reduced software quality and decreased developer productivity, overall leading to a more expensive and less competitive product.
To address the challenges, practitioners and researchers have developed a variety of strategies to manage, prevent and mitigate technical debt.
This section will provide an overview of the most common strategies, techniques and frameworks used in commercial environments to address technical debt.\\

\subsubsection{High-Level Mitigation Strategies}
To efficiently mitigate software decay and technical debt, proactive management strategies are essential. These strategies aim to prevent the accumulation of 
debt and address software entropy and decay directly by integrating quality assurance directly into everyday development process.
Such strategies include Agile methodologies like Scrum or \ac{XP} as well as technical practices
such as \ac{CI/CD}. These practices are designed to embed ongoing maintenance and quality assurance into routine workflows, thus combating software entropy at its core.\\
Agile methods emphasize frequent iterations, close collaboration between developers and stakeholders as well as continuos refactoring to prevent the
gradual degradation of the software quality and mitigation software decay.\\
Similarly, \ac{CI/CD} introduces rigorous automation, rapid feedback loops and early detection of defects to proactively controlling both technical debt accumulation
and broader software quality decay. Collectively, these methodologies create a culture of continuous improvement, adaptability and quality assurance, 
ensuring software maintainability and long-term project sustainability.\\
\paragraph{Agile Methodologies}
In their paper `Technical Debt Management in Agile Software Development: A Systematic Mapping Study' (2024)\footcite{leiteTechnicalDebtManagement2024} Leite et al.
investigated how agile methods can be used to manage technical debt. They found that `\ldots Scrum and Extreme Programming are the most utilized methodologies 
for managing technical debt.'\footcite[318]{leiteTechnicalDebtManagement2024} While this study focuses explicitly on technical debt, both Scrum and \ac{XP}
also inherently address the boarder issue of software decay by encouraging proactive quality management and continuous improvement practices.\\
Scrum was first presented by Ken Schwaber in his paper `SCRUM Development Process' (1995)\footcite{schwaberSCRUMDevelopmentProcess1997}. 
It has since become on of the most popular agile frameworks in the software industry. Scrum explicitly manages software quality and technical debt through iterative cycles called sprints.
After each sprint, the team reflects on their work, identifies quality issues, technical debt and potential decay indicators and plans improvements in
retrospectives. Leite et al. found that for identifying technical debt in Scrum the Sprint and Product Backlog are the most used artifacts.\footcite[315]{leiteTechnicalDebtManagement2024}
By explicitly managing these items with their workflow, teams effectively reduce both debt and software entropy, improving overall software maintainability.\\

\ac{XP}, first introduced by Kent Beck in his influential book `Extreme Programming Explained' (1999)\footcite{beckExtremeProgrammingExplained1999},
explictly integrates practices to enhance software quality and prevent software decay directly.
\ac{XP} practices such as pair programming, \ac{TDD} and \ac{CI} and continuos refactoring help maintain high software quality, thus preventing both debt accumulation
and broader software entropy.
Pair programming prevents decay by ensuring higher-quality code through collaborative review and knowledge sharing between developers.
Continuous Integration ensures regular, frequent code integration, significantly reducing integration complexity and associated decay risks.
\ac{TDD} ensures robust test coverage, catching defects early and preventing quality erosion.
Refactoring, a cornerstone XP practice, has proven its effectiveness empirically; for example, Moser et al.
demonstrated in their case study(2008)\footcite{moserCaseStudyImpact2008} that refactoring explicitly `prevents an explosion of complexity'\footcite[262]{moserCaseStudyImpact2008}
and promotes simpler, easier-to-maintain designs.
They found it drove developers toward simpler designs, reducing complexity, coupling, and long-term maintenance issues—directly counteracting software decay.\footcite[262]{moserCaseStudyImpact2008}
Beck argues that \ac{XP}'s incremental, continuos quality practices consistently maintain software quality and adaptability throughout development, directly addressing both debt and broader software decay.\\
Overall Agile methodologies, particularly Scrum and \ac{XP}, systematically manage technical debt and proactively prevent software decay by fostering continuous improvement, structured quality management and adaptability.
Complementing these Agile practices, the adoption of automated \ac{CI/CD} pipelines further enhances the proactive management of both technical debt
and broader software decay through rigorous quality control and systematic automation.\\

\paragraph{Continuous Integration/Continuous Deployment}
\ac{CI} was first introduced by Beck in the context of \ac{XP} and later refined by Martin Fowler in his influential article `Continuous Integration'\footcite{ContinuousIntegration}.
Fowler describes \ac{CI} as not only the frequent, automated integration of code into the main repository but also the systematic automation of building and testing process.
According to Fowler, `Self-testing code is so important to Continuous Integration that it is a necessary prerequisite.'\footcite{ContinuousIntegration}.
Furthermore, another critical prerequisite is `that they can correctly build the code,'\footcite{ContinuousIntegration} thus guaranteeing that code changes consistently
integrate without issues.\\

To further prevent technical debt and broader software decay, a quality analysis tools (e.g. static code analyzers such as SonarQube or DeepSource) are frequently integrated into
\ac{CI} pipelines. These tools often provide a metric to evaluate technical debt which is calculated based on the effort in minutes to fix the found maintainability issues.\footcite{UnderstandingMeasuresMetrics}
In their paper `Technical Debt Measurement during Software Development using Sonarqube: Literature Review and a Case Study' (2021)\footcite{murilloTechnicalDebtMeasurement2021}
Murillo et al. found that SonarQube is a useful tool for early debt detection. The estimated remediation effort metric allows for a good debt management prioritization.\footcite[5]{murilloTechnicalDebtMeasurement2021}
However during their research they noticed if they changed SonarQubes default rules by just 26 rules, the technical debt effort would increase from
1 hours and 50 minutes to 11 hours.\footcite[4]{murilloTechnicalDebtMeasurement2021} This and the fact that SonarQube can only detect code related debt and not for example
infrastructure or requirements debt, makes these tools useful but not a complete solution.\\

\ac{CD}, introduced by Jez Humble and David Farley in their foundational book `Continuous Delivery: Reliable Software Releases through Build, Test, and Deployment Automation' (2010)\footcite{humbleContinuousDeliveryReliable2010}
extends \ac{CI} by automating the entire software release pipeline. \ac{CD} ensures that the software is always in a releasable state.
According to Humble and Farley implementing a functional \ac{CD} pipeline `creates a release process that is repeatable, reliable, and predictable'\footcite[17]{humbleContinuousDeliveryReliable2010}.
Beyond predictability, additional significant benefits include team empowerment, deployment flexibility and substantial error reduction.
Specially, \ac{CD} effectively reduces errors, particularly those introduced by poor configuration management, including problematic areas such as
`configuration files, scripts to create databases and their schemas, build scripts, test harnesses, even development environments and operating system configurations'\footcite[19]{humbleContinuousDeliveryReliable2010}.\\

Empirical evidence from academic studies clearly demonstrates the effectiveness of \ac{CI/CD} in reducing both technical debt and software decay.
For instance, 
TODO: Find emperical evidence for CI/CD\\
\subsubsection{Specific Techniques and Tool-Supported Practices}
In addition to high-level strategies, there are a variety of specific techniques and tool-supported practices which can be used to managed technical debt 
and prevent software decay. These practices are often integrated into the development process to provide targeted, effective quality assurance and debt management.\\
\paragraph{Refactoring}
As mentioned earlier, refactoring is a key practice to prevent the code base from decaying. However, not only the complexity can be reduced by refactoring.
In their study `An Empirical Study of Refactoring Challenges and Benefits at Microsoft' (2014)\footcite{kimEmpiricalStudyRefactoringChallenges2014} Kim et al.
investigated benefits and challenges of refactoring at Microsoft, due to the fact that other studies showed very different results when it comes to the benefits of refactoring.



\subsubsection{Formal Debt Management and Tracking Techniques}
\subsubsection{Organizational and Cultural Factors}
\subsubsection{Conclusion}





