\section{Literature Review}
\subsection{Problems of Software Engineering}
Software has become an integral part of our daily lives. Certain expectations are made regarding the quality of software in terms of reliability, security and efficiency. These expectations come with challenges for the software developers across all industries.
To efficiently deal with these challenges, software engineers have tried for decades to develop methods and guides to overcome these issues. However in 1986 Frederick Brooks published his paper `No Silver Bullet' \footcite{brooksNoSilverBullet1987} 
in which he argues that: `\ldots building software will always be hard. There is inherently no silver bullet.' \footcite[3]{brooksNoSilverBullet1987}. He based this statement of the fact that there two types of difficulty in software development: the essential and the accidental.
The essential difficulties he names are complexity, conformity, changeability and invisibility. With complexity Brooks wants to describe the inherit intricacy of software systems: `Software entities are more complex for their size than perhaps any other human construct, because no two parts are alike'.\footcite[3]{brooksNoSilverBullet1987}
According to him this complexity makes `conceiving, describing, and testing them hard'\footcite[3]{brooksNoSilverBullet1987}.
The second essential Brooks names is conformity. To explain this he compares software development to physics. Even though they are similar complex, physics has the advantage on relying on a single set of laws or `creator'. The same cannot be said for software engineers. Brooks claims that
the complexity is of `arbitrary complexity, forced without rhyme or reason by the many human institutions and systems to which his interfaces must conform'\footcite[4]{brooksNoSilverBullet1987}. This is due to software being perceived as `the most comfortable'\footcite[4]{brooksNoSilverBullet1987} thing to change in a system.
Brooks explains the changeability issue by comparing software to other products like cars or computers. With these types of products, changes are difficult to make once the product is released. Software however is just `pure thought-stuff, infinitely malleable.'\footcite[4]{brooksNoSilverBullet1987} Another big part of the changeability issue
the fact that software often `survives beyond the normal life of the machine vehicle for which it is first written'\footcite[4]{brooksNoSilverBullet1987}. This means that software has to be adapted to new machines causing an extended life time of the software.
Invisibility is the last essential difficulty Brooks names. With this he means the difficulty to visualize software compared to other products. This makes the not only the creation difficult but also `severely hinders communication among minds'\footcite[4]{brooksNoSilverBullet1987}.
According to Brooks these issues are in `the very nature of software' \footcite[2]{brooksNoSilverBullet1987}. For him these difficulty are unlikely to be solved unlike the accidental difficulties.\\

In contrast the accidental difficulties arise from limitations of current languages, tools and methodologies. Brooks notes that these issues—such as inefficient programming environments, suboptimal development processes and integration challenges can be overcome as the industry improves its practices and technologies.
For example the adaptation of agile methodologies, integrated development environments and continuous integration have helped to overcome some of these accidental difficulties.\\

The persistent nature of these challenges Brooks presented have been since been substantiated by later empirical research. 