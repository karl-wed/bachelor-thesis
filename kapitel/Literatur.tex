\section{Literature Review}
\subsection{Problems of Software Engineering}
Software has become an integral part of our daily lives. Certain expectations are made regarding the quality of software in terms of reliability, security and efficiency. These expectations come with challenges for the software developers across all industries.
To efficiently deal with these challenges, software engineers have tried for decades to develop methods and guides to overcome these issues. However in 1986 Frederick Brooks published his paper `No Silver Bullet' \footcite{brooksNoSilverBullet1987} 
in which he argues that: `\ldots building software will always be hard. There is inherently no silver bullet.' \footcite[3]{brooksNoSilverBullet1987}. He based this statement of the fact that there two types of difficulty in software development: the essential and the accidental.
The essential difficulties he names are complexity, conformity, changeability and invisibility. With complexity Brooks wants to describe the inherit intricacy of software systems: `Software entities are more complex for their size than perhaps any other human construct, because no two parts are alike'.\footcite[3]{brooksNoSilverBullet1987}
According to him this complexity makes `conceiving, describing, and testing them hard'\footcite[3]{brooksNoSilverBullet1987}.
The second essential Brooks names is conformity. To explain this he compares software development to physics. Even though they are similar complex, physics has the advantage on relying on a single set of laws or `creator'. The same cannot be said for software engineers. Brooks claims that
the complexity is of `arbitrary complexity, forced without rhyme or reason by the many human institutions and systems to which his interfaces must conform'\footcite[4]{brooksNoSilverBullet1987}. This is due to software being perceived as `the most comfortable'\footcite[4]{brooksNoSilverBullet1987} thing to change in a system.
Brooks explains the changeability issue by comparing software to other products like cars or computers. With these types of products, changes are difficult to make once the product is released. Software however is just `pure thought-stuff, infinitely malleable.'\footcite[4]{brooksNoSilverBullet1987} Another big part of the changeability issue
the fact that software often `survives beyond the normal life of the machine vehicle for which it is first written'\footcite[4]{brooksNoSilverBullet1987}. This means that software has to be adapted to new machines causing an extended life time of the software.
Invisibility is the last essential difficulty Brooks names. With this he means the difficulty to visualize software compared to other products. This makes the not only the creation difficult but also `severely hinders communication among minds'\footcite[4]{brooksNoSilverBullet1987}.
According to Brooks these issues are in `the very nature of software' \footcite[2]{brooksNoSilverBullet1987}. For him these difficulty are unlikely to be solved unlike the accidental difficulties.\\

In contrast the accidental difficulties arise from limitations of current languages, tools and methodologies. Brooks notes that these issues—such as inefficient programming environments, suboptimal development processes and integration challenges can be overcome as the industry improves its practices and technologies.
For example the adaptation of agile methodologies, integrated development environments and continuous integration have helped to overcome some of these accidental difficulties.\\

The persistent nature of these challenges Brooks presented have been since been substantiated by later empirical research. For instance, Lehman and Ramil (2003)\footcite{lehmanSoftwareEvolutionBackground2003} discussed in their paper that software systems that are left unchecked will experience a decline in quality over time.
This phenomenon is encapsulated in Lehman's laws of software evolution, which he formulated in multiple papers. In their paper `Software evolution - Background, theory, practice' Lehman and Ramil present empirical observations that support the notion that software quality tends to deteriorate over time - a phenomenon often described als
software decay. 

\subsection{Software Decay and Technical Debt}
The term software decay was scientifically and statistically coined by Eick et al.\ in their paper `Does Code Decay? Assessing the Evidence from Change Management Data'(2001)\footcite{eickDoesCodeDecay2001}. They opened their paper by stating that `software does not age or "wear out" in the conventional sense.' \footcite[1]{eickDoesCodeDecay2001}
If nothing in the environment changes, the software could run forever. However, this is almost never the case as mainly two things change constantly: the hard- and software environments and the requirements of the software.
This is in accordance with the first two laws of Program Evolution Dynamics formulated by Belady and Lehman in 1976.
The first law states that `A system that is used undergoes continuing change until it is judged more cost effective to freeze and recreate it.'\footcite[228]{beladyModelLargeProgram1976}
The second law builds upon this with stating that `The entropy of a system (its unstructuredness) increases with time, unless specific work is executed to maintain or reduce it.'\footcite[228]{beladyModelLargeProgram1976}
Eick et al.\ found this to be true, as the data they processed proved `very strong evidence that code does decay.'\footcite[7]{eickDoesCodeDecay2001}
They base this on the fact that `the span of changes increases over time'\footcite[7]{eickDoesCodeDecay2001} as well as `the increase in span is accompanied by (and may cause) a breakdown in the modularity of the code.'\footcite[7]{eickDoesCodeDecay2001}
This increase of changes and therein resulting entropy result in code being `more difficult to change than it should be.'\footcite[3]{eickDoesCodeDecay2001}
Eick et al.\ based this on the following three criteria: cost of change, time to change and quality of the changed software. 