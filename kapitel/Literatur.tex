\section{Literature Review}
\subsection{Problems of Software Engineering}
Software has become an integral part of our daily lives. Certain expectations are made regarding the quality of software in terms of reliability, security and efficiency. These expectations come with challenges for the software developers across all industries.
To efficiently deal with these challenges, software engineers have tried for decades to develop methods and guides to overcome these issues. However in 1986 Frederick Brooks published his paper `No Silver Bullet' \footcite{brooksNoSilverBullet1987} 
in which he argues that: `\ldots building software will always be hard. There is inherently no silver bullet.' \footcite[3]{brooksNoSilverBullet1987}. He based this statement of the fact that there two types of difficulty in software development: the essential and the accidental.
The essential difficulties he names are complexity, conformity, changeability and invisibility. According to Brooks these issues are in `the very nature of software' \footcite[2]{brooksNoSilverBullet1987}. For him these difficulty are unlikely to be solved unlike the accidental difficulties
which have been and will be solved by the software engineering community. These accidental difficulties can be for example limitations of current tools and methods which can be overcome through time. 