\section*{Interviewfragebogen}
\textbf{Einleitung:}

Vielen Dank für Ihre Teilnahme. Ziel dieser Studie ist es, zu verstehen, wie in Ihrem Unternehmen aktuell Strategien zur Vermeidung und Minderung technischer Schuld und Softwareverfall (Software Decay) genutzt werden und wie effektiv diese Maßnahmen sind.

\subsection*{1. Allgemeine Informationen}

\begin{enumerate}
    \item Welche Rolle haben Sie aktuell inne und wie lange arbeiten Sie bereits im Bereich der Softwareentwicklung militärischer Projekte?
    \item Mit welchen Arten von Softwaresystemen arbeiten Sie aktuell hauptsächlich?
\end{enumerate}

\subsection*{2. Agile Methoden und technischer Schuld}

\begin{enumerate}[resume]
    \item Werden agile Methoden (z.B. Scrum, XP) in Ihren Projekten eingesetzt? Wenn ja, wie genau?
    \item Wie bewerten Sie die Effektivität agiler Methoden zur Prävention und Minderung technischer Schuld und Softwareverfall? Erläutern Sie bitte Ihre Erfahrungen.
    \item Falls agile Methoden nicht möglich sind, wie geht Ihr Team dann mit technischer Schuld um? Welche konkreten Praktiken nutzen Sie?
\end{enumerate}

\subsection*{3. CI/CD-Praktiken}

\begin{enumerate}[resume]
    \item Nutzen Sie aktuell CI/CD? Falls ja, beschreiben Sie kurz Ihre CI/CD-Implementierung (z.B. Tools, Abläufe).
    \item Wie wirksam schätzen Sie CI/CD zur Reduktion technischer Schuld oder Softwareverfall ein?
    \item Welche Faktoren erschweren aktuell die Einführung oder optimale Nutzung von CI/CD in Ihren Projekten?
\end{enumerate}

\subsection*{4. Code Reviews}

\begin{enumerate}[resume]
    \item Werden regelmäßig Code Reviews durchgeführt? Falls ja, wie genau (z.B. Peer Review, automatisierte Tools)?
    \item Wie effektiv sind Code Reviews aus Ihrer Sicht, um technische Schuld oder Softwareverfall zu reduzieren? Nennen Sie gerne konkrete Erfahrungen.
    \item Welche Faktoren könnten die Effektivität von Code Reviews weiter verbessern?
\end{enumerate}

\subsection*{5. Architekturmanagement und Konformität}

\begin{enumerate}[resume]
    \item Wie wird in Ihren Projekten die Softwarearchitektur aktuell verwaltet und dokumentiert?
    \item Wie stellen Sie sicher, dass die Implementierung der geplanten Architektur entspricht? Nutzen Sie Tools oder automatisierte Checks?
    \item Was sind die größten Herausforderungen im Bereich Architekturkonformität, speziell bei langlaufenden oder Legacy-Projekten?
\end{enumerate}

\subsection*{6. Umgang mit Legacy-Systemen und langen Lebenszyklen}

\begin{enumerate}[resume]
    \item Welche besonderen Herausforderungen bestehen bei der Wartung und dem Management technischer Schuld in Legacy-Systemen oder Projekten mit besonders langen Laufzeiten?
    \item Welche Strategien oder Maßnahmen haben sich in Ihrem Team konkret bewährt, um Softwareverfall und technische Schuld bei Legacy-Systemen zu reduzieren?
\end{enumerate}

\subsection*{7. Bewertung und Effektivität der aktuellen Mitigationsstrategien}

\begin{enumerate}[resume]
    \item Welche konkreten Strategien oder Maßnahmen setzen Sie aktuell hauptsächlich ein, um technische Schuld und Softwareverfall aktiv zu verhindern oder zu verringern?
    \item Wie bewerten Sie insgesamt die Effektivität dieser Strategien?
    \item Welche Hindernisse oder Herausforderungen erschweren aktuell die erfolgreiche Umsetzung Ihrer Mitigationsstrategien?
    \item Gibt es Arten technischer Schuld, die besonders schwierig zu handhaben sind? Falls ja, warum?
    \item Können Sie ein konkretes Beispiel für eine erfolgreiche Mitigationsmaßnahme nennen? Was genau machte diese Maßnahme erfolgreich?
    \item Welchen Einfluss haben externe Faktoren (z.B. regulatorische Vorgaben, Sicherheitsanforderungen, Beschaffungsprozesse) auf Ihre Fähigkeit, technische Schuld effektiv zu managen?
\end{enumerate}

\subsection*{Abschluss}

\begin{enumerate}[resume]
    \item Welche Maßnahme wäre Ihrer Erfahrung nach besonders wirksam, um technische Schuld und Softwareverfall in Ihrer Organisation zukünftig effektiver zu managen?
    \item Gibt es weitere relevante Erfahrungen oder Einsichten, die Sie mitteilen möchten?
\end{enumerate}

Vielen Dank für Ihre Teilnahme!
