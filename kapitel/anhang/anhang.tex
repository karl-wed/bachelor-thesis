
\subsection*{Interviewleitfaden – Bachelorarbeit Softwareverfall}

\subsubsection*{1. Einstieg und Hintergrund}

\begin{enumerate}
  \item Welche Rolle haben Sie aktuell im Bereich der Softwareentwicklung militärischer Systeme? Seit wann sind Sie in diesem Umfeld tätig und wie sieht Ihre typische Projektarbeit aus?
  \item Mit welchen Arten von Softwaresystemen arbeiten Sie hauptsächlich? (z.\,B. eingebettete Systeme, einsatzkritische Plattformen, Verwaltungssoftware, Legacy-Systeme)
\end{enumerate}

\subsubsection*{2. Legacy-Systeme und lange Lebenszyklen}

\begin{enumerate}[resume]
  \item Welche Herausforderungen begegnen Ihnen beim Umgang mit Legacy-Systemen oder langlaufenden Projekten?
  \item Kommt es im Projektverlauf regelmäßig zu Problemen, die durch veraltete Strukturen oder historisch gewachsene Systeme entstehen?
  \item Welche Auswirkungen haben lange Lebenszyklen auf Wartbarkeit und Weiterentwicklung?
\end{enumerate}

\subsubsection*{3. Strategien zur Qualitätssicherung}

\begin{enumerate}[resume]
  \item Welche Maßnahmen setzen Sie ein, um Softwareverfall oder technische Schuld zu vermeiden oder zu reduzieren? (z.\,B. Refactoring, Reviews, Automatisierung) Sehen Sie diese Maßnahmen als sinnvoll?
  \item Gibt es Bedingungen, unter denen technische Schuld bewusst in Kauf genommen wird? Wird bewusste Schuld getrackt?
\end{enumerate}

\subsubsection*{4. Agile Methodik}

\begin{enumerate}[resume]
  \item Verwenden Sie agile Methodiken wie Scrum oder XP? Falls ja, wie wirken sich diese Praktiken auf den Umgang mit technischer Schuld und Softwareverfall aus?
  \item Wenn keine agilen Methodiken verwendet werden, wie werden Refactoring und Schuldenabbau sonst durchgeführt?
\end{enumerate}

\subsubsection*{5. Architekturmanagement}

\begin{enumerate}[resume]
  \item Wie wird die Systemarchitektur geplant, dokumentiert und gepflegt?
  \item Wie wird sichergestellt, dass sich die Umsetzung an der vorgesehenen Architektur orientiert, insbesondere bei älteren Projekten?
  \item Welche Herausforderungen treten bei der Weiterentwicklung von Architekturentscheidungen auf?
\end{enumerate}

\subsubsection*{6. Automatisierung und CI/CD}

\begin{enumerate}[resume]
  \item Setzen Sie CI/CD in Ihren Projekten ein? Wie sehen die Prozesse konkret aus und hilft CI/CD gegen die zuvor genannten Probleme?
  \item Welche Werkzeuge und Metriken verwenden Sie zur Qualitätssicherung im Build- oder Deployprozess? Was sind Ihre Erfahrungen mit diesen Werkzeugen? Gab es bestimmte Stärken oder Schwächen, die sich auf die Codequalität oder Wartbarkeit ausgewirkt haben?
\end{enumerate}

\subsubsection*{7. Reviews und Tests}

\begin{enumerate}[resume]
  \item Wie wird bei Ihnen mit Code Reviews gearbeitet? Wer beteiligt sich, und wie wird der Prozess organisiert?
  \item Welche Rolle spielen Tests? Nutzen Sie automatisierte Verfahren wie Unit-Tests oder Test-Driven Development? Gibt es bestimmte Tests, die Ihrer Meinung nach am besten zur Vermeidung von technischen Schulden helfen?
\end{enumerate}

\subsubsection*{8. Besondere Rahmenbedingungen im militärischen Umfeld}

\begin{enumerate}[resume]
  \item Welche Anforderungen an Sicherheit, Zuverlässigkeit oder regulatorische Vorgaben wirken sich auf Ihre Arbeit aus?
  \item Inwiefern beeinflussen Beschaffungsprozesse oder lange Freigabeschleifen Ihre Projekte in Bezug auf technische Schulden und Softwareverfall?
\end{enumerate}

\subsubsection*{9. Abschluss}

\begin{enumerate}[resume]
  \item Können Sie ein Beispiel nennen, in dem eine Maßnahme zur Reduktion von technischer Schuld oder zur Vermeidung von Softwareverfall besonders erfolgreich war? Was genau machte diese Maßnahme aus Ihrer Sicht erfolgreich?
  \item Gibt es weitere Erfahrungen oder Empfehlungen, die Sie im Zusammenhang mit Softwareverfall oder technischer Schuld teilen möchten?
\end{enumerate}


Vielen Dank für Ihre Teilnahme!
