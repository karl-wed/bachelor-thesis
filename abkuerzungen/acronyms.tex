
\section*{\langde{Abkürzungsverzeichnis}\langen{List of Abbreviations}}

\begin{acronym}[WYSIWYG]\itemsep0pt %der Parameter in Klammern sollte die längste Abkürzung sein. Damit wird der Abstand zwischen Abkürzung und Übersetzung festgelegt
  \acro{WYSIWYG}{What you see is what you get}
  \acro{BSI}{Federal Office for Information Security}
  \acro{CI}{Continuous Integration}
  \acro{CI/CD}{Continuous Integration/Continuous Delivery}
  \acro{CD}{Continuous Delivery}
  \acro{CPM}{Customer Product Management}
  \acro{DOD}{Department of Defense}
  \acro{EVB-IT}{Ergänzende Vertragsbedingung für die Beschaffung von IT-Leistungen}
  \acro{NRC}{National Research Council}
  \acro{QA}{Quality Assurance}
  \acro{SASPF}{Integrated Standard-Application-Software-Product Family}
  \acro{TDD}{Test Driven Development}
  \acro{UI}{User Interface}
  \acro{VS-NFD}{Verschlusssache - Nur für den Dienstgebrauch}
  \acro{XP}{Extreme Programming}
\end{acronym}
